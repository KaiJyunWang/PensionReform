\documentclass[12pt]{article}
\usepackage{CJKutf8}
\usepackage{amsmath, amssymb, amsthm}
\usepackage{accents}
\usepackage{graphicx}
\usepackage{dsfont}
\usepackage[natbibapa]{apacite}
\usepackage{geometry}
\geometry{
 a4paper,
 total={210mm,297mm}, 
 left=25.4mm,
 right=25.4mm,
 top=25.4mm,
 bottom=25.4mm
}
\usepackage{lipsum}
\usepackage{setspace}
\usepackage{booktabs,tabularx}
\usepackage[bitstream-charter]{mathdesign}
\usepackage[T1]{fontenc}
\usepackage{hyperref}
\usepackage{xcolor}
\hypersetup{
    colorlinks,
    linkcolor= blue,
    citecolor= blue,
    urlcolor= blue
}
\usepackage{comment}

\newcommand{\abs}[1]{\left|#1\right|}


\title{Understanding the Effects of Pension Reforms: A 
Structural Model Approach}
\author{Kai-Jyun Wang}
\date{\today}

\begin{document}
\setstretch{1.2}
\begin{CJK*}{UTF8}{bsmi}
\maketitle

\section{Introduction}

In recent years, due to factors such as extended life 
expectancy and declining birth rates, pension systems in 
countries around the world have faced pressure for reform 
and Taiwan is no exception. As a consequence, the effect of 
pension reforms are widely investigated in many countries. 
However, to our knowledge, one of the major pension systems 
in Taiwan, named as Labor Insurance, has never been studied. 
What is the effect of the pension reforms of the Labor 
Insurance on the retirement decision and the private 
savings? And, particularly, how does the crowd respond to 
the information of the reforms?

These questions are in particular important under the 
institution after 2009, when the institution has been 
changed so that anyone who had participated in the Labor 
Insurance and continued after 2009 could choose to retire 
with two types of the pension schemes, a lump-sum scheme 
and a monthly pension scheme. For example, consider a person 
encountering an unpredictable benefit reduction and a 
predicted one. The former may induce a large shift in both 
saving and labor supply, while the later may only respond 
with a moderate or tiny shift. The later can even choose to 
retire earlier or shift his choice from a monthly pension 
scheme to a lump-sum scheme before the actual reform comes, 
lest he suffers from the possibly incoming reforms. 

To answer these questions, we construct a life-cycle model 
of households choosing to participate the labor market, the 
amount of savings, and updating their believes on the chance 
of facing a reform. The model incorporates the 
heterogeneities in the discount factor, the preference of 
leisure and the prior believes of the occurrence of the 
reform. Our model also captures the complex structure of the 
pension schemes.

\section{Background}

In 2009, Labor Insurance, one of the major pension system in 
Taiwan, transitioned to a pension system adopting the defined 
benefit system. After the transition, the workers who had 
attended the Labor Insurance before 2009 can choose to follow 
the new rule or the old one while those had not can only 
follow the new one. 

Before the transition in 2009, the conditions to determine 
whether one can receive the benefit is quite complicated. 
One can only get paid if one of the followings is satisfied: 
(i) Male working for at least 1 year and aging 60 (55 for 
females). (ii) Working for at least 15 years and aging 55. 
(iii) Working at the same company for at least 25 years. 
(iv) Working for at least 25 years and aging 50.

A worker being eligible for the benefit would receive a 
lump-sum transfer form the government when retiring. The 
amount depends on the working years and the average monthly 
insurance salary\footnote{The average monthly insurance 
salary is calculated by the grades of Labor Insurance 
salary, which ranges from 27,470 NTD to 45,800 NTD in 2024 
and is adjusted annually by the administration.} in three 
years before the retirement. The benefit formula is
\begin{equation}
    \text{pb}_t = \text{AMIS}_t\times
    \max\{\text{WY}_t,\ 2\text{WY}_t-15\},
\end{equation}
where $\text{pb}_t$ is the pension benefit, $\text{AMIS}_t$ 
is the average monthly insurance salary and $\text{WY}_t$ is 
the working years.

After 2009, the conditions for applying the benefit was 
canceled and the rule became a dichotomy that one may 
follows the defined benefit pension rule or the lump-sum 
payment similar to the one before 2009. To access the pension 
rule, one needs to work for at least 15 years. The benefit 
formula for the lump-sum rule and the monthly pension rule is 
\begin{equation}\label{eq:pblumpsum}
    \text{pb}_t = \text{AMIS}_t\times
    \max\{\text{WY}_t,\ 2\text{WY}_t-15\},
\end{equation}
\begin{equation}\label{eq:pbmonthly}
    \text{pb}_t = 
    \max\{\text{AMIS}_t\times\text{WY}_t\times0.775\%+3000,\ 
    \text{AMIS}_t\times\text{WY}_t\times 1.55\%\},
\end{equation}
respectively. Also, $\text{AMIS}_t$ is calculated with the 
highest 60 months' wage in the career.

Additionally, the pension benefits are influenced by the 
standard retirement age. Individuals who have accrued at 
least 15 years of working experience and opt to claim 
benefits before reaching the standard retirement age will 
encounter a reduction of 4\% for each year prior to this 
benchmark. Conversely, those who choose to postpone 
retirement beyond the standard age will see their benefits 
increase by 4\% per year. Eligibility to claim benefits 
spans a window of five years before and after the standard 
retirement age. Since 2018, the standard retirement age has 
been gradually increasing by one year for every two years.

\section{Related Literature}

Our research is connected to various papers focusing on 
pension, retirement, and saving. The pension system typically 
exerts two potentially offsetting effects on individuals' 
saving behavior. On one hand, pension wealth may directly 
substitute personal wealth; however, conversely, the pension 
system often provides individuals incentives to delay 
retirement, thereby extending the period of asset 
accumulation and increasing saving. Consequently, the impact 
of pension reform on retirement and private savings is of 
significant importance and the related literature is 
discussed in this chapter. Additionally, the research 
conducted in Taiwan is presented towards the end of this 
chapter.

In early work, \cite{feldstein1974} indicates that the Social 
Security pension crowded out 30 to 50 percent of personal 
saving using the aggregate time series data from the United 
States. Subsequently, \cite{attanasio2003} observes that the 
reduction in pension wealth leads to an increase in the 
saving rate in Italy's pension reform in 1992, where the 
offset is about 0.35. \cite{aguila2011} shows that the shift 
from the pay-as-you-go system to personal retirement account 
system in Mexico's 1997 reform increases the pension wealth 
and crowds the private savings for the lower-income worker. 
In the mean time, some recent studies suggest that the 
crowd-out effect is minimal or even nonexistent. For 
instance, \cite{feng2011} examines the reform in China from 
1995 to 1997 and reports a relatively low offsetting effect 
ranging from 0.1 to 0.16. Similarly, \cite{chetty2014} 
discovers no substitution between private savings and public 
pensions following Denmark's 1999 reduction in retirement 
pension subsidies. In another study, \cite{lacowska2018} 
identifies a moderate crowd-out effect of 0.3 resulting from 
Poland's 1999 pension reform. 

In another related branch concerning retirement decisions, 
early studies such as \cite{burtless1986} and 
\cite{gustman1986} delve into the explanation of the two-peak 
phenomenon in retirement age. They attribute this phenomenon 
to the heterogeneity of workers' preferences and the varying 
returns of Social Security for each retirement age. Building 
on this foundation, \cite{rust1997} developed a model where 
agents are unable to engage in intertemporal substitution. 
This work highlights that the incompleteness of the medical 
insurance market may compel workers to prolong their careers 
until getting access to Medicare. \cite{french2011} further 
confirms the results in an estimated life-cycle model 
consisting of labor and savings decisions. \cite{haan2014} 
calculated that in order to relieve the fiscal burden due to 
the increasing life expectancy, either postponing the 
retirement threshold age by 3.76 years or reduce the benefit 
by 26.8\% is sufficient. \cite{daminato2023} estimates a 
life-cycle model incorporating savings, portfolio choice 
and retirement through a quasi-experimental variation from 
pension reforms in Italy for identification. The model 
predicts a significant social security wealth effect on 
retirement. 

For the research conducted in Taiwan, \cite{yang2009} uses 
data from the Manpower Utilization Survey to investigate the 
impact of changes in old and new Labor Pension systems on 
wages. The results show that, overall, there is no 
significant change in wages within two years of the 
implementation of the new and old systems. However, for 
workers who only enter the workforce after the changes, 
their wages decrease significantly, roughly in proportion to 
the employer contribution rate in the new system. 
\cite{reform2017} calibrates a general equilibrium model to 
investigates the reforms encompassing Civil Servant and 
Teacher Insurance and Labor Insurance. The simulation result 
shows that after the reforms, overall employment levels 
declined. The decrease in real wages results in the reduced 
labor supply, leading to an increase in the overall 
unemployment rate. \cite{cheng2020} also conducts a 
counterfactual analysis using a calibrated general 
equilibrium model, finding that the pension debt can be 
balanced if the government increases income taxes by 5.4\%, 
increases consumption taxes by 6.2\%, increases pension taxes 
by 9.3\%, or decreases the pension benefits by 23.5\%.

\section{The Model}
Consider an individual who aims at maximizing his expected 
lifetime utility at age \(t = 40,\ldots,T\), where \(T\) is 
one's maximum lifespan. These individuals maximize their 
utility by choosing their consumption \(c\), labor supply 
\(n\) (in extensive margin), pension type \(p\) and their 
bequests \(b\) to their descendants. At period \(t\), the 
utility flow is 
\begin{equation}
    u(c_t,l_t) = \frac{1}{1-\gamma}(c_t^\eta l_t^{1-\eta})^{1-\gamma},
\end{equation}
where \(l_t\) is leisure. We allow \(\eta\) to vary across 
individuals. For individuals with higher \(\eta\), they value 
their consumption more comparing to leisure.

Leisure is normalized and determined by
\begin{equation}
    l_t = 1 - \frac{260}{364}n_t.
\end{equation}

When a worker dies, he obtains utility from leaving his 
remaining assets, \(a_t\), as bequests. The functional form 
follows \cite{de-nardi2004}.
\begin{equation}
    b(a_t) = \frac{\theta_b}{1-\gamma}(\kappa+a_t)^{1-\gamma}.
\end{equation}

The individual faces survival risk depends on his age. 
Following \cite{thatcher1999}, we model their mortality 
function \(\mu\) as
\begin{equation}
    \mu_t = \mu(t) = \frac{\theta_1}{1+\theta_2\exp(-\theta_3t)} + \theta_4.
\end{equation}
Note that we determine the maximum lifespan through the law 
of mortality. That is, \(T = \min\{t\mid \mu(t) \geq 1\}\). 

The worker's productivity is
\begin{align}
    \log(w_{it}) &= \max\{X_{it}\delta + f_i + \epsilon_{it},\ \log(\underaccent{\bar}{w})\},\\
    \epsilon_{it} &= \rho\epsilon_{i,t-1} + \xi_{it},\label{eq:permanent-income-shock}
\end{align}
where \(X_{it}\) includes age and education, \(f_i\) captures 
the unobserved heterogeneity of productivity across 
individuals and \(\xi_{it}\overset{iid}{\sim}N(0,\sigma_\xi^2)\). 
Besides, the Labor Standards Act sets the minimum wage, 
\(\underaccent{\bar}{w}_t\); hence the real wage that one 
receives is the maximum between the real productivity and 
the minimum wage \(\underaccent{\bar}{w}\).

Also, since the Labor Standards Act allows the employers to 
retire the workers aging 65, we assume that if the real 
productivity of a worker is lower than 
\(\underaccent{\bar}{w}\) and the worker ages 65, he is 
retired by the employers. To simplify the model, once the 
worker stops working after 65, he cannot return to work. 

Whenever an individual is eligible for the pension, he can 
choose between two pension plans as (\ref{eq:pblumpsum}) or 
(\ref{eq:pbmonthly}) and retire. We denote his pension status 
as \(p_t\), following
\begin{equation}
    p_{t+1}\in
    \begin{cases}
        \{1,2\},\quad & p_t=1, \\
        \{4\},\quad & p_t=2,4, \\
        \{3\},\quad &p_t=3\text{ or }p_t=1,\text{WY}_t\geq15.
    \end{cases}
\end{equation}
Thus 
\begin{equation}
    \text{pb}_t = 
    \begin{cases}
        0, &\quad p_t=1,4,\\
        \text{AMIS}_t\times\max\{\text{WY}_t,\ 2\text{WY}_t-15\}, &\quad p_t=2,\\
        \max\{\text{AMIS}_t\times\text{WY}_t\times0.775\%+3000,\ \text{AMIS}_t\times\text{WY}_t\times 1.55\%\}, &\quad p_t=3.
    \end{cases}
\end{equation}

The law of motion of the \(\text{AMIS}_t\) is defined as 
\begin{equation}
    \text{AMIS}_{t+1} = 
    \begin{cases}
        \frac{\text{WY}_t}{\text{WY}_t+n_t}\text{AMIS}_t + \frac{n_t\text{CMIS}_t}{\text{WY}_t+n_t}, \quad &\text{WY}_t<5,\\
        \text{AMIS}_t + \frac{1}{5}\max\{0,\text{CMIS}_tn_t-\text{LMIS}_t\}, \quad &\text{WY}_t\geq5,
    \end{cases}
\end{equation}
where \(\text{CMIS}_t\) is a stair function of the wage with 
a ceiling and a floor, \(\text{LMIS}_t\) is the lowest salary 
among past 5 years and \(\text{WY}_t\) is the work year. That 
is,
\begin{align}
    \text{LMIS}_{t+1} &= \max\{\text{LMIS}_t,\text{CMIS}_t\},\\
    \text{WY}_{t+1} &= \text{WY}_t + n_t.
\end{align}

An individual faces several budget constraints,
\begin{align}
    c_t + a_{t+1} &= (1+r)a_t + \text{pb}_t + w_tn_t - 0.2\times\text{CMIS}_tn_t\tau - \phi_l\mathds{1}\{n_t-n_{t-1}=1\},\label{eq:asset-accumulation}\\
    a_{t+1} &\geq 0,\label{eq:no-borrow}\\
    c_t &\geq \underaccent{\bar}{c}\label{eq:cfloor},
\end{align}
where \(\phi_l\) is the transition cost of starting to work 
and \(\underaccent{\bar}{c}\) is the consumption floor. We 
assume that an individual cannot borrow against his future 
income including wage and pension. 

In addition, the individual in our model also considers the 
uncertainty of reforms. We simplify this uncertainty to be a 
binary case, either reformed pension scheme (\(R_t = 1\)) and 
unreformed scheme (\(R_t = 0\)), where the reformed scheme 
can be in several types such as reducing the benefit, 
increasing the standard retirement age, or increasing the 
pension tax. We assume that there are two types of 
governments, either a high chance (\(p_H\)) or a low chance 
(\(p_L\)) to reform. Let \(\pi_t\) be the prior of the reform 
occurring next period. We assume that the individuals update 
\(\pi_t\) in a Bayesian manner according to the information 
that he receives. That is, we assume that the news is either 
positive or negative, following a Bernoulli distribution that 
is heterogeneous in different types of governments. The 
updating rule is 
\begin{equation}\label{eq:update}
    \begin{split}
        \pi_{t+1} 
        &= \frac{\Pr(z_t, R_t\mid H)\pi_t}{\Pr(z_t, R_t\mid H)\pi_t + \Pr(z_t, R_t\mid L)(1-\pi_t)} \\
        &= \frac{\Pr(z_t\mid R_t,H)\Pr(R_t\mid H)\pi_t}{\Pr(z_t\mid R_t,H)\Pr(R_t\mid H)\pi_t + \Pr(z_t\mid R_t,L)\Pr(R_t\mid L)(1-\pi_t)}.
    \end{split}
\end{equation}
where \(z_t\) stands for the independent and identically 
distributed drawn sequence of news in period \(t\). The reformed 
state is treated as an absorbing state. 

Let \(\beta\) be the time preference, we may write the 
individual's problem recursively as
\begin{equation}
    \begin{split}
        V_t(x_t,R_t) = \max_{c_t,n_t,p_t}& \biggl\{u(c_t,1-\frac{260}{364}n_t) + \beta\mu_tb(a_{t+1}) \\
        &+ \beta(1-\mu_t)\biggl[\pi_t\int V_{t+1}(x_{t+1},R_{t+1}=1)\:dF(x_{t+1}\mid x_t,c_t,n_t,p_t,t) \\
        &+ (1-\pi_t)\int V_{t+1}(x_{t+1},R_{t+1}=0)\:dF(x_{t+1}\mid x_t,c_t,n_t,p_t,t)\biggr] \biggr\}
    \end{split}
\end{equation}
subject to equations (\ref{eq:no-borrow}) and 
(\ref{eq:cfloor}). We also allow the time preference and the 
prior of reform to vary across individuals. The vector 
\(x_t=\left(a_t,n_{t-1},p_{t-1},\text{WY}_t, \text{AMIS}_t,
\text{LMIS}_t,\epsilon_t\right)\) contains the state variables 
whose law of motions follow equations 
(\ref{eq:permanent-income-shock}) to 
(\ref{eq:asset-accumulation}) and (\ref{eq:update}). The 
solution of the model consists of the saving rules, work 
rules and pension application rules. The model is solved 
numerically using value function iteration. Linear 
interpolations are used in iterations in order to approximate 
the continuous state variables.


\section{Estimation}
Due to the complexity of the model, we adopt a two-step 
approach. In the first step, we estimate the wage equation 
and the mortality law outside of the main model. To estimate 
the main model, we use the simulated method of moments (SMM) 
approach. The method consists of two loops, the inner loop 
and the outer loop. In the inner loop, we fix the estimated 
parameters and solve the policy function under the parameters 
to obtain the simulated moments. Next, in the outer loop, the 
parameters are updated to minimize the distance between the 
simulated moments and the real moments. 

\subsection{Wage Equation} 
The main issue in estimating the wage equation is the 
selection bias problem. 

\subsection{Mortality Law} 
The mortality parameters $\theta = (\theta_1, \theta_2, 
\theta_3, \theta_4)$ are estimated by the maximum liklihood 
estimation. The likelihood function is 
\begin{equation}
    \mathcal{L}(\theta) = \prod_{i=1}^N\prod_{t=40}^{T_i}\mu_t^{d_{it}}(1-\mu_t)^{1-d_{it}},
\end{equation}
where \(d_{it}\) is the indicator of the individual \(i\)'s 
survival status at age \(t\). 

\subsection{Reform Signal}
For the reform signal distributions of different types of 
governments, we scrawl the news and use machine learning 
method to rate the tendency of the news. After that, we 
estimate the parameters of the distributions by the 
expectation maximization algorithm. 

\subsection{SMM Target Moments}
To identify the parameters in the main model, we select the 
following moments to be matched. Also, we use the optimal 
weighting matrix to improve the efficiency of the estimation. 


\section{Results}

\section{Policy Experiments} 

\section{Conclusion}

\bibliographystyle{apacite}
\bibliography{ref_res_app}
\end{CJK*}
\end{document}
